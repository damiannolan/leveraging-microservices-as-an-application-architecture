\documentclass[journal]{IEEEtran}

\usepackage[sorting=none]{biblatex}
\addbibresource{bibliography.bib}

% correct bad hyphenation here
\hyphenation{op-tical net-works semi-conduc-tor}


\begin{document}
% Title ---------------------------------------------
\title{Microservices \\ As an Application Architecture}


\author{Damian Nolan}

% make the title area
\maketitle

% Abstract ------------------------------------------
\begin{abstract}
Microservice Architecture is an approach in software engineering and design that aims to divide an application into small constituents that are highly cohesive and loosely coupled. The following paper is intended to give a relatively high level overview and exploration into the world of Microservices that can be useful for both experts and novices alike. Microservice Architecture is continuously evolving and gaining traction throughout the world of software design - Throughout this paper we will introduce Microservice Architecture as a concept, compare and contrast Microservices with traditional Monolithic Architecture and review Docker, as a containerization platform for building Microservice applications.
\end{abstract}

\section{Introduction}
% The very first letter is a 2 line initial drop letter followed
% by the rest of the first word in caps.
% 
% form to use if the first word consists of a single letter:
% \IEEEPARstart{A}{demo} file is ....
% 
% form to use if you need the single drop letter followed by
% normal text (unknown if ever used by IEEE):
% \IEEEPARstart{A}{}demo file is ....
% 
% Some journals put the first two words in caps:
% \IEEEPARstart{T}{his demo} file is ....
% 
% Here we have the typical use of a "T" for an initial drop letter
% and "HIS" in caps to complete the first word.
\IEEEPARstart{T}{he} term "Microservice" does not hold a true or set definition. It was merely a term introduced by a number of software architects at a workshop near Venice in May, 2011 to give a context to a number of reoccurring design principles and patterns that seemed to be emerging more frequently.
	Just under a year later, James Lewis presented at 33rd Degree in Krakow where he spoke about building systems composed of systems and focusing on the Unix philosophy of small and simple. \cite{JamesLewis33rdDegree} 
And so, with the introduction of a hot new buzzword on the scene, "Microservices" began to get more and more attention. 

Martin Fowler describes the Microservice Architectural style as a approach to developing a single application as a suite of independently deployable services. \cite{MicroservicesResourceGuide} Each service should be independent in its our right and provide functionality based around a single responsibility. Services should be loosely coupled, provide high cohesion and adhere to the single responsibility principle. All of a sudden we can see that many of the traits we mentioned are key concepts in Object Orientated Programming. This is common in design principles of Service Orientated Architecture. Stubbings and Polovina \cite{StubbingsPolovina} explore and contrast the how Object Orientated expertise can be leveraged in the design of Service Orientated Architecture (SOA).
% Finish introduction / polish it off.


% Talk about martin fowler definition
% Talk about SOA vs Microservices - netflix
% Characteristics of MSA
% Monolith vs MSA
% Docker

\section{Are Microservices just SOA?}

This immediately sparks the debate - Is Microservice Architecture really just SOA? Upon the rise of the term "Microservices" into the community, many of SOA architects were coming forward and saying "We've been doing this for years!". And as Martin Fowler mentions \cite{GOTOConference}, Service Orientated Architecture is a very broad term. It encompasses a vast landscape of design concepts, principles, patterns and implementation standards. Fowler goes on to mention that in essence Microservices can be really be considered a subset of SOA.

Indeed, this is a popular and well defined statement as it is backed up by Adrian Cockcroft, the man responsible for pioneering the architectural style at Netflix. He describes the architecture as "fine-grained SOA", in his keynote at Dockercon in 2014. \cite{adriancockcroft} 

% Conclusion -------------------------------------------
\section{Conclusion}
The conclusion goes here.

% References -------------------------------------------
% trigger a \newpage just before the given reference
\newpage
\printbibliography

% that's all folks
\end{document}
